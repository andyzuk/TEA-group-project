% Options for packages loaded elsewhere
\PassOptionsToPackage{unicode}{hyperref}
\PassOptionsToPackage{hyphens}{url}
%
\documentclass[
]{article}
\usepackage{lmodern}
\usepackage{amssymb,amsmath}
\usepackage{ifxetex,ifluatex}
\ifnum 0\ifxetex 1\fi\ifluatex 1\fi=0 % if pdftex
  \usepackage[T1]{fontenc}
  \usepackage[utf8]{inputenc}
  \usepackage{textcomp} % provide euro and other symbols
\else % if luatex or xetex
  \usepackage{unicode-math}
  \defaultfontfeatures{Scale=MatchLowercase}
  \defaultfontfeatures[\rmfamily]{Ligatures=TeX,Scale=1}
\fi
% Use upquote if available, for straight quotes in verbatim environments
\IfFileExists{upquote.sty}{\usepackage{upquote}}{}
\IfFileExists{microtype.sty}{% use microtype if available
  \usepackage[]{microtype}
  \UseMicrotypeSet[protrusion]{basicmath} % disable protrusion for tt fonts
}{}
\makeatletter
\@ifundefined{KOMAClassName}{% if non-KOMA class
  \IfFileExists{parskip.sty}{%
    \usepackage{parskip}
  }{% else
    \setlength{\parindent}{0pt}
    \setlength{\parskip}{6pt plus 2pt minus 1pt}}
}{% if KOMA class
  \KOMAoptions{parskip=half}}
\makeatother
\usepackage{xcolor}
\IfFileExists{xurl.sty}{\usepackage{xurl}}{} % add URL line breaks if available
\IfFileExists{bookmark.sty}{\usepackage{bookmark}}{\usepackage{hyperref}}
\hypersetup{
  pdftitle={Untitled},
  pdfauthor={andy zukerberg},
  hidelinks,
  pdfcreator={LaTeX via pandoc}}
\urlstyle{same} % disable monospaced font for URLs
\usepackage[margin=1in]{geometry}
\usepackage{longtable,booktabs}
% Correct order of tables after \paragraph or \subparagraph
\usepackage{etoolbox}
\makeatletter
\patchcmd\longtable{\par}{\if@noskipsec\mbox{}\fi\par}{}{}
\makeatother
% Allow footnotes in longtable head/foot
\IfFileExists{footnotehyper.sty}{\usepackage{footnotehyper}}{\usepackage{footnote}}
\makesavenoteenv{longtable}
\usepackage{graphicx,grffile}
\makeatletter
\def\maxwidth{\ifdim\Gin@nat@width>\linewidth\linewidth\else\Gin@nat@width\fi}
\def\maxheight{\ifdim\Gin@nat@height>\textheight\textheight\else\Gin@nat@height\fi}
\makeatother
% Scale images if necessary, so that they will not overflow the page
% margins by default, and it is still possible to overwrite the defaults
% using explicit options in \includegraphics[width, height, ...]{}
\setkeys{Gin}{width=\maxwidth,height=\maxheight,keepaspectratio}
% Set default figure placement to htbp
\makeatletter
\def\fps@figure{htbp}
\makeatother
\setlength{\emergencystretch}{3em} % prevent overfull lines
\providecommand{\tightlist}{%
  \setlength{\itemsep}{0pt}\setlength{\parskip}{0pt}}
\setcounter{secnumdepth}{-\maxdimen} % remove section numbering

\title{Untitled}
\author{andy zukerberg}
\date{1/10/2020}

\begin{document}
\maketitle

Literature spanning decades clearly shows that parental involvement at
school and home has a positive impact on a child's educational and
physical development. Parental involvement can be divided into
activities at home that support schooling and those that take place at
school. In this project we will look at the factors that impact a
parent's participation directly in school activities. These activities
include volunteering in the school, participating in a fundraiser,
attending general school meetings (e.g.~open house), attending Parent
and Teacher Association (PTA) meetings, serving on a school committee,
meeting with a guidance counselor, going to a regularly scheduled parent
teacher conference or attending a school or class event (play, dance,
sports). We will look at the role of parent involvement at home, income
level, race/ethnicity, school type (public/private) and household size
in predicting parent involvement in school activities and meetings
attended.

Numerous studies show the beneficial effects of parent involvement in
their child's education. For this reason, schools and organizations like
the National PTA work to increase parent participation. In reviewing the
literature on parent involvement, Cotton and Reed concluded ``The
research overwhelmingly demonstrates that parent involvement in
children's learning is positively related to achievement. Further, the
research shows that the more intensively parents are involved in their
children's learning, the more beneficial are the achievement effects.
This holds true for all types of parent involvement in children's
learning and for all types and ages of students.'' Studies have
demonstrated this impact in many subject areas and across different
demographic groups (add cites) As a result of this strong evidence, The
National PTA has developed a set of standards for engaging parents (see:
\url{https://s3.amazonaws.com/rdcms-pta/files/production/public/National_Standards_Implementation_Guide_2009.pdf}).
As part of the standards, schools are encouraged to find ways to involve
parents in schools. However, not all parents participate in their
children's education or participate in the same ways. In this report, we
will look at the predictors of parents who participate in their child's
education.

\hypertarget{methodology}{%
\section{\texorpdfstring{\textbf{Methodology}}{Methodology}}\label{methodology}}

\hypertarget{research-questions}{%
\subsection{\texorpdfstring{\textbf{\emph{Research
Questions}}}{Research Questions}}\label{research-questions}}

The purpose of this study is to investigate the demographic and social
factors related to parental involvement in schools. We believe that
certain demographic variables such as parental education, income, and
school satisfaction have a positive relationship with parental
involvement in school-based activities while student age has a negative
relationship. Our research questions are as follows:

\begin{enumerate}
\def\labelenumi{\arabic{enumi}.}
\tightlist
\item
  Are the questions related to parental involvement of educational
  activities at home and school unidimensional? If not, what is the
  internal structure of these items?
\item
  What factors are related to parental involvement in school-based
  activities?
\end{enumerate}

\hypertarget{survey-data-and-sample}{%
\subsection{\texorpdfstring{\textbf{\emph{Survey data and
sample}}}{Survey data and sample}}\label{survey-data-and-sample}}

The data came from the National Household Education Survey (NHES)
conducted by the National Center for Education Statistics (NCES) in
2016. More specifically, the Parent and Family Involvement in Education
(PFI) questionnaire. The PFI questionnaire is a nationally
representative questionnaire that measures parental involvement in
educational activities, including at home and school. Parents and
guardians with children enrolled in grades K-12 are sampled. The 2016
sample size is 14,075. Our analysis focuses on a subset of the
respondents who were not homeschooled. The final dataset consisted of
13,523 parents' responses to survey questions and demographic
information. The sample was mostly white (76\%). \textbf{ADD OTHER
DESCRIPTIVES RELATED TO SAMPLE DEMOGRAPHICS}. On average parents expects
their children to earn a bachelor's degree. The median household income
is between \$60K and 70K. Parents have a bachelor's degree on average
(median).

\begin{longtable}[]{@{}lclllll@{}}
\toprule
Parental Expectation & \textless HS & HS & TS & Assoc & BA/BS &
Professional\tabularnewline
\midrule
\endhead
Value & 1 & 2 & 3 & 4 & 5 & 6\tabularnewline
Observations & & & & & &\tabularnewline
\bottomrule
\end{longtable}

\hypertarget{measures}{%
\subsection{\texorpdfstring{\textbf{\emph{Measures}}}{Measures}}\label{measures}}

Two major constructs utilized for this study were parents' satisfaction
with the school their child attended and parental involvement in
educational activities at school and home. Parent involvement in school
was measured using 8 dichotomous (yes/no) items. School Satisfaction was
measured using 5 items from the same survey. Parents were asked to rate
their overall satisfaction with the school their child(ren) were
enrolled using a 4-point Likert scale (1: not satisfied - 4 highly
satisfied). These items were 4 category Likert Scale. Educational
support was measured by several dichotomous items related to parental
involvement at school and home.

\hypertarget{analysis}{%
\subsection{\texorpdfstring{\textbf{\emph{Analysis}}}{Analysis}}\label{analysis}}

Several analyses will be conducted to answer the research questions.
There were 27 items used to measure parental support of educational
development at school and home and parents' satisfaction with school on
the PFI. We hypothesized that these 27 items represented three factors,
school satisfaction, parental involvement at school, and parental
involvement in educational related activities at home. To determine
which items were similar, we conducted an exploratory factor analysis
(EFA) with maximum likelihood to determine the internal structure.
Therefore, the EFA will address the first research question. This
research question will help to determine which items can be considered a
scale for the regression analysis. Second, descriptive analyses will be
conducted to describe the data, once the scales have been created. Last,
a multiple regression model will be conducted to help explain the
factors related to parental involvement school-based activities.

\hypertarget{results}{%
\section{\texorpdfstring{\textbf{Results}}{Results}}\label{results}}

add stuff here

\hypertarget{exploratory-factor-analysis}{%
\subsection{\texorpdfstring{\textbf{\emph{Exploratory Factor
Analysis}}}{Exploratory Factor Analysis}}\label{exploratory-factor-analysis}}

The EFA supported a 3-factor structure with 18 items. The items were
chosen based on their factor loading. If they had a factor loading
greater than 0.30 then it was considered related. A fair number of items
have low communalities, indicating their they do not share much variance
with other items on the questionnaire. However, many of the factor
loadings, strength of the relationship between the factor and item were
greater than the cutoff. The fit statistics, the Tucker Lewis Index,
RMSEA, were appropriate to suggest good fit. The three-factor solution
explained 88\% of the variance between the items. Therefore, the factor
analysis was deemed to be suitable.

Nine items were eliminated because they failed to meet a minimum
criterion a simple structure. A simple structure is one in which the
primary factor loading is greater than .30 (really .4), and no-cross
loadings greater than 0.3. The factors were Since the items are related,
we can create three subscales, parental involvement in school-based
activities, parental involvement in educational activities at homes, and
satisfaction with school. The school satisfaction factor has 5 items
(\(\alpha\) = .90), the parent support at school factor has 7 items
(\(\alpha\) = .74) and the family support at home has 6 items
(\(\alpha\) = .71). For the purposes of research an alpha value of 0.70
is good. Composite scores or summed score will be created for these
three subscales for use in subsequent analyses.

\hypertarget{descriptive-analysis-correlations-and-such}{%
\subsection{\texorpdfstring{\textbf{\emph{Descriptive Analysis
(Correlations and
Such)}}}{Descriptive Analysis (Correlations and Such)}}\label{descriptive-analysis-correlations-and-such}}

add stuff here.

\hypertarget{regression-analysis}{%
\subsection{\texorpdfstring{\textbf{Regression
Analysis}}{Regression Analysis}}\label{regression-analysis}}

add stuff here.

\hypertarget{discussion-and-conclusions}{%
\section{\texorpdfstring{\textbf{Discussion and
Conclusions}}{Discussion and Conclusions}}\label{discussion-and-conclusions}}

\end{document}
